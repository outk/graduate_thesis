\documentclass[11pt,a4j,notitlepage]{jreport}

\usepackage{times}
\usepackage[dvipdfmx]{graphicx}	%図を表示するのに必要
\usepackage[dvipdfmx]{color}	%jpgなどを表示するのに必要
\usepackage{amsmath,amssymb}	%数学記号を出すのに必要
\usepackage{setspace}
\usepackage{bm}
\usepackage{braket}	%ブラケットを表示するのに必要
\usepackage{otf}
\usepackage{here}	%図を好きな位置に表示
\usepackage[subrefformat=parens]{subcaption}	%サブキャプション

%PDFの機能(しおり機能、ハイパーリンク機能)が使えるようにする
%しおりの文字化けを防ぐ
\usepackage{atbegshi}
\AtBeginShipoutFirst{\special{pdf:tounicode 90ms-RKSJ-UCS2}}
%hyperrefのverが2007-06-14 6.76i以前の時は↓
%\AtBeginShipoutFirst{\special{pdf:tounicode 90ms-RKSJ-UCS2}}
\usepackage[dvipdfmx,bookmarkstype=toc,colorlinks=true,urlcolor=blue,linkcolor=blue,
citecolor=blue,linktocpage=true,bookmarks=true,setpagesize=false,
pdftitle={量子状態トモグラフィー},
pdfauthor={小林哲也},%
pdfsubject={Bachelor's thesis in 2020}]{hyperref}
\usepackage[numbers,sort]{natbib}
\usepackage{tocbibind}%目次、表一覧、図一覧をしおりに入れる
 
%式、図、表番号の付け方の再定義
\makeatletter
	\renewcommand{\theequation}{%
	\thesection.\arabic{equation}}
	\@addtoreset{equation}{section}
	\def\thefigure{\thesection.\arabic{figure}}
	\@addtoreset{figure}{section}
	\renewcommand{\thetable}{%
	\thesection.\arabic{table}}
	\@addtoreset{table}{section}
\makeatother

%本文と図の余白
\setlength\intextsep{30pt} 
 
\renewcommand\bibname{参考文献}	%関連図書の表示を参考文献に変更
\newcommand{\fig}[1]{図~\ref{#1}}	%図の引用の再定義
\newcommand{\tab}[1]{表~\ref{#1}}	%表の引用の再定義
\newcommand{\eq}[1]{式~\eqref{#1}}	%式の引用の再定義
 
%大きなフォントの定義(表紙用)
\def\HUGE{\fontsize{32pt}{36pt}\selectfont} %\fontsize{フォントの大きさ}{baselineskip}
 
 
%本文
\begin{document}

	%タイトル
	\begin{titlepage}
		\begin{center}\begin{LARGE}
			\vspace{1em}
			{令和元年度}\\
			\vspace{1.5em}
			{卒業研究}\vspace{3em}\\
			\textbf{\HUGE 量子状態トモグラフィー}\\
			\vspace{4em}
			{\LARGE 指導教員}\\
			\vspace{0.8em}
			{\Huge\bf 山本 俊 教授}\\
			\vspace{0.2\vsize}
			{大阪大学 基礎工学部\\ 電子物理科学科 物性物理科学コース\\ 山本研究室\\学籍番号 09D16031}\\
			\vspace{0.8em}
			{\Huge\bf 小林哲也}\\
			\vspace{3em}
			{\Large 2020年3月6日}
		\end{LARGE}\end{center}
	\end{titlepage}

	%目次のページだけローマ数字に設定
	\pagenumbering{roman}

	%目次サブセクションまで表記
	\setcounter{tocdepth}{2}
	\tableofcontents

	\clearpage

	%図目次
	\listoffigures

	%ページ数をリセットしアラビア数字に変更
	\clearpage
	\pagenumbering{arabic}


	\chapter{序論}
	\section{研究の背景と目的}
	近年注目を増している量子情報、量子コンピューティング研究・開発において実験的に生成された量子状態を正確に認識することは極めて重要である。しかしながら、量子的制約により、量子状態を直接観測することは現時点では不可能である。そこで、量子状態トモグラフィーが用いられる。量子状態トモグラフィーとは、同じ量子状態を複数生成し、それぞれを測定基底を変えて測定をすることで得られた観測データから、最尤推定などを用いて真の量子状態を推定することである。本論文では、量子状態トモグラフィーの一般的理論を説明した後、実験的誤りに対する耐性があるアルゴリズムを紹介する。最後に、これらのアルゴリズムを実装し確認する。


	\chapter{理論}
	\section{量子状態トモグラフィーの一般的理論}


	\subsection{密度行列}

	量子状態は密度行列によって表される。密度行列$\hat{\rho}$とは以下の性質を満たす行列である。
	\begin{equation}
		Tr[\hat{\rho}]=1
		\label{eq2.1}
	\end{equation}
	また、密度行列の固有値を$\lambda$とすると、
	\begin{equation}
		0 \leq \lambda \leq 1, \ \ \ \  ^\forall \lambda
		\label{eq2.2}
	\end{equation}
	$d$次元の密度行列を$\hat{\rho}_d$と表記する。



	\subsection{1 qubitトモグラフィー (直交基底)}

	qubitは$2$次元の密度行列で表される。

	1 qubitトモグラフィーのために、パウリ演算子を導入する。パウリ演算子は単位演算子$I$と$SU(2)$の$X,\ Y,\ Z$からなる。
	\begin{equation}
		\begin{aligned}
			I \equiv \hat{\lambda}_0 = \begin{pmatrix} 1 & 0 \\ 0 & 1 \end{pmatrix}&,\ \ X \equiv \hat{\lambda}_1 = \begin{pmatrix} 0 & 1 \\ 1 & 0 \end{pmatrix}\\
			Y \equiv \hat{\lambda}_2 = \begin{pmatrix} 0 & -i \\ i & 0 \end{pmatrix}&,\ \ Z \equiv \hat{\lambda}_3 = \begin{pmatrix} 1 & 0 \\ 0 & -1 \end{pmatrix}
		\end{aligned}
		\label{eq2.3}
	\end{equation}
	これらを用いて、1 qubitの密度行列$\hat{\rho}_2$は次のように表される。
	\begin{equation}
		\hat{\rho}_2 = \frac{1}{2} \sum_{j=0}^3 r_j \hat{\lambda}_j, \ \ \ \ r_j \in Re
		\label{eq2.4}
	\end{equation}
	$SU(2)$はトレースが0なので、密度行列$\hat{\rho}_2$は規格化のために$r_0 = 1$を満たす必要がある。また、ほかのパラメータ$r_{j=1,...,3}$は
	\begin{equation}
		r_1^2 + r_2^2 + r_3^2 \leq 1
		\label{eq2.5}
	\end{equation}
	のみ満たす。また、$r_j$は$r_j = Tr[\hat{\rho}_2 \hat{\lambda}_j]$で得られる。\\
	
	したがって、1 qubitの密度行列は
	\begin{equation}
		\hat{\rho}_2 = \frac{1}{2} \begin{pmatrix}
			1 + \braket{\hat{\lambda}_3} &
			\braket{\hat{\lambda}_1} - i \braket{\hat{\lambda}_2} \\
			\braket{\hat{\lambda}_1} + i \braket{\hat{\lambda}_2} &
			1 - \braket{\hat{\lambda}_3}
			\end{pmatrix}
	\end{equation}
	で表される。
	
	上式から、1 qubitの密度行列は3つの測定だけで求まりそうだが、実験的には4つ目の基底$\hat{\lambda}_0$の測定によって密度行列の規格化が必要である。また、$\langle \hat{\lambda}_j \rangle$の値によっては、$(2.1.2)$式を満たさないことがある。したがって、最尤推定などを用いて物理的に意味のある密度行列を見つける必要がある。

	$SU(2)$生成子は必ず物理的状態を示すとは限らないが、$\hat{\lambda}_{0,...,3}$は常に物理的に意味のある状態の密度行列の線形和で表すことができる。例えば、パウリ演算子は光学系での測定としては物理的意味がない。しかし、光子の測定を偏向基底で行う際に以下のようなものが用いられる。
	\begin{equation}
		\begin{aligned}
			\ket{H} \bra{H} = \frac{1}{2}[\hat{\lambda}_0 + \hat{\lambda}_3] \ket{V} \bra{V} = \frac{1}{2}[\hat{\lambda}_0 - \hat{\lambda}_3]\\
			\ket{D} \bra{D} = \frac{1}{2}[\hat{\lambda}_0 + \hat{\lambda}_1] \ket{L} \bra{L} = \frac{1}{2}[\hat{\lambda}_0 + \hat{\lambda}_2]
		\end{aligned}
	\end{equation}
	ここで、$\ket{H} = \ket{0}, \ \ket{V} = \ket{1}, \ \ket{D} = (\ket{0} + \ket{1}) / \sqrt{2} , \ \ket{L} = (\ket{0} + i\ket{1}) / \sqrt{2}$

	このようにどの直交した測定基底を選んでも他のいくつかの測定演算子$\hat{\Pi}_k$を用いて$\hat{\lambda}_j = \frac{1}{2} \sum_{k} a_{jk} \hat{\Pi}_k$と表すことができる。そして、トモグラフィーは測定結果$a_{jk} = \langle \hat{\Pi}_k \rangle = Tr[\hat{\rho}_2 \hat{\Pi}_k]$を測定することで行われる。

	\subsection{1 qubitトモグラフィー (非直交基底)}

	実際には、測定は測定装置側の基底を偏向せずに量子状態を変化させて測定する。そのため、1 qubitの状態を測定基底に合わせて、$| 0 \rangle + | 1 \rangle$や$| 0 \rangle + i | 1 \rangle$から$| 0 \rangle$へ量子状態を変化させることが難しい場合がある。その場合、測定基底を$| 0 \rangle$と
	\begin{equation}
		\begin{aligned}
			\ket{\theta_+} &= \frac{1}{\sqrt{2}} [\cos \theta \ket{0} + \sin \theta \ket{1}]\\
			\ket{\varphi_+} &= \frac{1}{\sqrt{2}} [\cos \varphi \ket{0} + i \sin \varphi \ket{1}]
		\end{aligned}
	\end{equation}
	とすることができる。$\theta, \varphi$は小さい値でもよい。つまり、1 qubitトモグラフィーはある測定基底と少しの摂動があれば行える。実験系によってはこれは重要になることがある。

	任意の基底$\ket{\psi_\nu}$での測定は射影子$\hat{\lambda}_\nu = \ket{\psi_\nu} \bra{\psi_\nu}$で表され、これらの基底による観測回数$n_\nu$は
	\begin{equation}
		n_\nu = \mathcal{N} \braket{\psi_\nu | \hat{\rho} | \psi_\nu}
	\end{equation}
	で表される。($\mathcal{N}$は定数)

	\subsection{Quditへの拡張}

	まず、$SU(d)$を準備する。($d$次元の$SU$群)

	$d$次元の要素行列$\{ e_j^k|k,j=1,...,d \}$は
	\begin{equation}
		\big( e_j^k \big) _{\mu \nu} = \delta_{\nu j} \delta_{\mu k}, \ \ \ \ 1 \leq \nu, \mu \leq d
	\end{equation}
	で表され、$\mu$行$\nu$列目の要素が1で他の要素すべてが0である行列である。

	これらの行列は交換関係を満たす。
	\begin{equation}
		\big[ e_j^i, e_l^k \big] = \delta_{kj} e_l^i - \delta_{il} e_j^k
	\end{equation}
	$d(d-1)$個のトレースが0の行列が存在する。
	\begin{equation}
		\begin{aligned}
			\Theta_j^k = e_j^k + e_k^j, \ \ \beta_j^k = -i \big( e_j^k - e_k^j \big) \ \ \ \ \ \ \ \ 1 \leq k < j \leq d
		\end{aligned}
	\end{equation}
	これらは$SU(d)$群の非対角生成子である。

	対角生成子として残り$d-1$個のトレースが0の行列を
	\begin{equation}
		\eta_r^r = \sqrt{\frac{2}{r(r-1)}} \Biggl[ \sum_{j=1}^r e_j^j - r e_{r+1}^{r+1} \Biggr]
	\end{equation}
	とすると、これで$d^2-1$個の生成子が得られる。\\
	ここで$\lambda$行列を次のように定義する。
	\begin{equation}
		\begin{aligned}
			\hat{\lambda}_{(j-1)^2 + 2(k-1)} &= \Theta_j^k\\
			\hat{\lambda}_{(j-1)^2 + 2k-1} &= \beta_j^k\\
			\hat{\lambda}_{j^2-1} &= \eta_{j-1}^{k-1}
		\end{aligned}
	\end{equation}
	$d$次元に拡張してもこれらの形式は完全エルミート演算子基底である。つまり、$\hat{\lambda}$がエルミートかつ次式を満たす。
	\begin{equation}
		\sum_{j=0}^{d^2 - 1} \hat{\lambda}_j = \hat{1}
	\end{equation}

	$d$次元に拡張しても$(2.1.4)$式はそのまま適用できる。つまり、密度行列$\hat{\rho}_d$は生成子の線形結合で表される。
	\begin{equation}
		\hat{\rho}_d = \frac{1}{d} \sum_{j=0}^{d^2 - 1} r_j \hat{\lambda}_j
	\end{equation}
	これは1 quditの密度行列である。規格化のために係数$r_0$は1とし、$Tr \big[ \hat{\rho}_d^2 \big] \leq 1$より$\sum_{j=1}^{d^2 - 1} r_j^2 \leq d(d-1)/2$を満たす。

	\subsection{Multi quditsへの拡張}

	Multi qubitsでは、演算子のヒルベルト空間を規格化された単位行列$\hat{\lambda}_0$を含んだ$SU(2)$を生成子のテンソル積で定義する。
	\begin{equation}
		SU(2) \otimes SU(2) \otimes ・・・ \otimes SU(2)
	\end{equation}

	2 quditsでは$d^2$の次元を持った密度行列$\hat{\rho}_{2 d}$は同様に拡張できる。\\
	$\hat{\lambda}_0$を含む$\hat{\lambda}$行列のテンソル積$\hat{\lambda}_{j 1} \otimes \hat{\lambda}_{j 2}$のすべての組はそれぞれ線形独立なので、$\hat{\rho}_{2 d}$は次のように表される。
	\begin{equation}
		\hat{\rho}_{2 d} = \frac{1}{d^2} \sum_{j1,j2=0}^{d^2 - 1} r_{j 1, j 2} \hat{\lambda}_{j 1} \otimes \hat{\lambda}_{j 2}
	\end{equation}
	同様に、n quditsでは
	\begin{equation}
		\hat{\rho}_{n d} = \frac{1}{d^n} \sum_{j1,...,jn=0}^{d^2 - 1} r_{j 1,...,j n} \hat{\lambda}_{j 1} \otimes ・・・ \otimes \hat{\lambda}_{j n}
	\end{equation}

	\subsection{密度行列の再構成}

	簡単のために$\hat{\Gamma}_\nu = \hat{\lambda}_{j 1} \otimes ・・・ \otimes \hat{\lambda}_{j n}$とすると、密度行列は
	\begin{equation}
		\hat{\rho}_{n d} = \sum_{\nu = 0}^{d^n - 1} \tilde{r}_\nu \hat{\Gamma}_\nu
	\end{equation}
	で表される。$\tilde{r}_\nu$は$d^n$要素あるベクトルの$\nu$番目の要素で
	\begin{equation}
		\tilde{r}_\nu = Tr \big[ \hat{\Gamma}_\nu \hat{\rho}_{n d} \big]
	\end{equation}
	これを$(2.1.9)$式に代入して
	\begin{equation}
		n_\nu = \mathcal{N} \sum_{\mu = 0}^{d^n - 1} B_{\nu, \mu} \tilde{r}_\nu
	\end{equation}
	ここで、$B_{\nu, \mu}$は$d^n \times d^n$行列の$\nu$行$\mu$列番目の要素で
	\begin{equation}
		B_{\nu, \mu} = \braket{\psi_\nu | \hat{\Gamma}_\mu | \psi_\nu}
	\end{equation}
	$B_{\nu, \mu}$が可逆行列であれば
	\begin{equation}
		\tilde{r}_\nu = \mathcal{N}^{-1} \sum_{\nu = 0}^{d^n - 1} \hat{M}_\nu n_\nu = \sum_{\nu = 0}^{d^n - 1} \hat{M}_\nu s_\nu
	\end{equation}
	ここで、$\hat{M}_\nu$は$d \times d$行列で
	\begin{equation}
		\hat{M}_\nu = \sum_{\nu = 0}^{d^n - 1} \big( B^{-1} \big)_{\nu, \mu} \hat{\Gamma}_\nu
	\end{equation}
	$\hat{M}_\nu$の性質から
	\begin{equation}
		\sum_\nu Tr \big[ \hat{M}_\nu \big] \ket{\psi_\nu} \bra{\psi_\nu} \hat{\rho}_{n d} = \hat{\rho}_{n d}
	\end{equation}
	両辺でトレースをとると
	\begin{equation}
		\sum_\nu Tr \big[ \hat{M}_\nu \big] n_\nu = \mathcal{N}
	\end{equation}
	したがって、任意の密度行列$\hat{\rho}_{n d}$は次のように再構成される。
	\begin{equation}
		\hat{\rho}_{n d} = \frac{\sum_\nu \hat{M}_\nu n_\nu}{\sum_\nu Tr \big[ \hat{M}_\nu \big] n_\nu} 
	\end{equation}


	\section{最尤推定}

	これで密度行列は実験の測定基底と観測回数によって一意に求まるが、これまでで求まる密度行列は必ずしも密度行列の性質をすべて満たしているとは限らない。

	この問題を避けるために最尤推定を利用する。手順は以下の通りである。

	\begin{enumerate}
		\item \underline{密度行列の性質を満たす密度行列を生成する}
		\item \underline{1で求めた密度行列が非物理的であれば尤度関数を導入し、次の最尤推定を行う}
		\item \underline{Iterativeなアルゴリズムを用いて尤度関数を最大化させもっとも確からしい密度行列を求める}
	\end{enumerate}


	\subsection{密度行列の確認}

	まず、密度行列の性質を満たす行列を生成する。

	半正定値行列$\hat{G}$は次の式を満たすとする。
	\begin{equation}
		\braket{\psi | \hat{G} | \psi} \geq 0,\ \ \ \ \ ^\forall \ket{\psi}
	\end{equation}
	$\hat{G} = \hat{T}^\dagger \hat{T}$と書けるどんな行列$\hat{G}$も必ず半正定値行列となる。実際、
	\begin{equation}
		\braket{\psi | \hat{T}^\dagger \hat{T} | \psi} = \braket{\psi' | \psi'} \geq 0
	\end{equation}
	ここで、$\ket{\psi'} = \hat{T} \ket{\psi}$である。さらに、$(\hat{T}^\dagger \hat{T})^\dagger = \hat{T}^\dagger (\hat{T}^\dagger)^\dagger = \hat{T}^\dagger \hat{T}$、すなわち$\hat{G}$はエルミートである。

	規格化のために
	\begin{equation}
		\hat{g} = \frac{\hat{T}^\dagger \hat{T}}{Tr \big[ \hat{T}^\dagger \hat{T} \big] } 
	\end{equation}
	とすると、$\hat{g}$は密度行列の数学的条件をすべて満たす。

	この性質を利用するために、$\hat{T}$を$d^2$個の実数変数$t$を用いて、
	\begin{equation}
		\hat{T}_{(t)} = \begin{pmatrix}
			t_1 & 0 & & \cdots & 0 \\
			t_{d+1} + it_{d+2} & t_2 & & & \\
			\vdots & & \ddots &  & \vdots \\
			& & & t_{d-1} & 0 \\
			t_{d^2 - d + 1} + it_{d^2 - d + 2} & & \cdots & t_{d^2 - 1} + it_{d^2} & t_d
		\end{pmatrix}
	\end{equation}
	とすると、
	\begin{equation}
		\hat{\rho}_p = \frac{\hat{T}^\dagger_{(t)} \hat{T}_{(t)}}{Tr \big[ \hat{T}^\dagger_{(t)} \hat{T}_{(t)} \big] }
	\end{equation}
	は明らかに密度行列の性質を持つ。
	つまり、この複素行列$\hat{T}_{(t)}$を実験値から適切に求めることで最も確からしい密度行列が求められる。

	そこで、$(2.1.28)$式で実験値から求めた密度行列が$(2.2.5)$式となっているかを確認する。
	$(2.2.5)$式は複素数のCholesky分解そのものであるので、確認するのは容易である。(Appendix A)
	ここで求めた複素行列$\hat{T}_{(t)}$が上述の条件を満たしていれば、以降の最尤推定は必要ない。

	最尤推定が必要な場合、得られた密度行列は非物理的な値になっているので、最尤推定の初期値としてそのまま利用することはできない。
	実験的なデータ誤差や偏りに左右されないように最大混合状態$\hat{I}$を最尤推定の初期値とすることは合理的である。
	しかし、今回は真の密度行列をある程度推測してから最尤推定を行うことで計算時間を短縮できる仮定のもとに、初期値を実験データを利用して求めた密度行列にする場合も考える。
	つまり、$(2.2.4)$式の制約から対角項を実数のみ利用し求めた密度行列を初期値とする。
	また、計算途中で0の除算が発生した場合は0の代わりに微小項を代入する。

	\subsection{$\hat{R}\hat{\rho}\hat{R}アルゴリズム$}

	まず、尤度関数$\mathcal{L} (\hat{\rho})$を導入する。

	一般の測定$\hat{\Pi}$はPOVMで表されるので、$Tr \big[ \hat{\Pi}_i \hat{\rho} \big] \geq 0$を満たす。
	ここで、総測定回数を$N$、それぞれの測定基底$\hat{\Pi}_i$における測定回数を$f_i$とする。
	量子状態$\hat{\rho}$におけるある測定回数の集合{$f_i$}の尤度関数は$\mathcal{L} (\hat{\rho}) = \prod_i \Pr_i^{f_i}$で得られる。
	$\Pr_i = Tr \big[ \hat{\Pi}_i \hat{\rho} \big]$はそれぞれの基底で得られる確率である。
	最終的な目標はこの尤度関数を最大化させる密度行列$\hat{\rho}_0$を見つけることである。
	ここで相対頻度を$\tilde{f}_i = \frac{f_i}{N}$とする。
	$\ket{y_i}$は正規直交基底として、測定基底を$\hat{\Pi}_i = \ket{y_i} \bra{y_i}$とする。
	また、POVMは完全関係を満たすので、
	\begin{equation}
		\sum_i \ket{y_i} \bra{y_i} = \hat{I}
	\end{equation}

	\subsection*{\underline{尤度関数の増加}}

	対数尤度関数は$\log \mathcal{L} (\hat{\rho}) = \sum_i \tilde{f}_i \log \Pr_i$である。ここで、
	\begin{equation}
		\hat{R} = \sum_i \frac{\tilde{f}_i}{\Pr_i} \ket{y_i} \bra{y_i}
	\end{equation}
	とすると、$\hat{R} \hat{\rho}$の対数尤度関数は
	\begin{equation}
		\begin{aligned}
			\log \mathcal{L} \big( \hat{R} \hat{\rho}) &= \sum_i \tilde{f}_i \log \Big( Tr \big[ \hat{\Pi}_i \hat{R} \hat{\rho} \big] \Big) \\
			&=  \sum_i \tilde{f}_i \log \Bigg( Tr \Big[ \hat{\Pi}_i \sum_j \frac{\tilde{f}_j}{\Pr_j} \ket{y_j} \bra{y_j} \hat{\rho} \Big] \Big) \\
			&= \sum_i \tilde{f}_i \log \Bigg( \sum_j \frac{\tilde{f}_j}{\Pr_j} \braket{y_i | y_j} \braket{y_j | \hat{\rho} | y_i} \Bigg) \\
			&= \sum_i \tilde{f}_i \log \tilde{f}_i
		\end{aligned}
	\end{equation}
	となる。したがって、$\hat{R}$による密度行列の更新前後の対数尤度関数の差は
	\begin{equation}
		\log \mathcal{L} \big( \hat{R} \hat{\rho} \big) - \log \mathcal{L} \big( \hat{\rho} \big) = \sum_i \tilde{f}_i \log \tilde{f}_i - \sum_i \tilde{f}_i \log \Pr_i = \sum_i \tilde{f}_i \log \frac{\tilde{f}_i}{\Pr_i}
	\end{equation}
	ここで、$\sum_i \tilde{f}_i = \sum_i \Pr_i = 1,\ \ 0 < \tilde{f},\ \ \Pr_i < 1$なので、Jensenの不等式
	\begin{equation}
		\prod_i \Big[ \frac{x_i}{a_i} \Big]^{\tilde{f}_i} \leq \sum_i \tilde{f}_i \frac{x_i}{a_i} \ \ \ \ \Big( \sum_i x_i = 1,\ x_i \geq 0,\ a_i > 0 \Big)
	\end{equation}
	を用いると、
	\begin{equation}
		\begin{aligned}
			- \sum_i \tilde{f}_i \log \frac{\tilde{f}_i}{\Pr_i} &= \sum_i \tilde{f}_i \log \frac{\Pr_i}{\tilde{f}_i} \\
			&= \log \Bigg( \prod_i \Big( \frac{\Pr_i}{\tilde{f}_i} \Big)^{\tilde{f}_i} \Bigg) \\
			&\leq \log \Big( \sum_i \tilde{f}_i \frac{\Pr_i}{\tilde{f}_i} \Big) \\
			&= \log 1 = 0
		\end{aligned}
	\end{equation}
	したがって、$\hat{\rho}^{(k+1)} = \hat{R} (\hat{\rho}^{(k)}) \hat{\rho}^{(k)}$と密度行列を更新していけば尤度関数は必ず増加する。

	\subsection*{\underline{尤度関数の収束性}}

	Jensenの不等式から
	\begin{equation}
		\prod_i \Big[ \frac{x_i}{a_i} \Big]^{\tilde{f}_i} \leq \sum_i \tilde{f}_i \frac{x_i}{a_i} \Longleftrightarrow \prod_i x_i^{\tilde{f}_i} \leq \prod_i a_i^{\tilde{f}_i} \sum_k \tilde{f}_k \frac{x_k}{a_k}
	\end{equation}
	より、尤度関数は
	\begin{equation}
		\begin{aligned}
			\mathcal{L} (\hat{\rho}) &= \prod_i Tr \big[ \ket{y_i} \bra{y_i} \hat{\rho} \big]^{\tilde{f}_i} \\
			&\leq \prod_i a_i^{\tilde{f}_i} \sum_k \tilde{f}_k \frac{Tr [ \ket{y_i} \bra{y_i} \hat{\rho}]}{a_k} \\
			&= \prod_i a_i^{\tilde{f}_i} Tr \big[ \hat{\rho} \hat{R} (a) \big]
		\end{aligned}
	\end{equation}
	\begin{equation}
		\hat{R} (a) = \sum_i \frac{\tilde{f}_i}{a_i} \ket{y_i} \bra{y_i}
	\end{equation}
	ここで、任意の半正定値演算子$\hat{R} = \sum_i \lambda_i \ket{y_i} \bra{y_i},\ \lambda_i \geq 0$の最大固有値を$\max_i \lambda_i$とすると、
	\begin{equation}
		Tr [\hat{\rho} \hat{R}] \leq \max_i \lambda_i
	\end{equation}
	よって、$\hat{R} (a)$の最大固有値を$\lambda (y, a)$として、
	\begin{equation}
		\mathcal{L} (\hat{\rho}) \leq \lambda (y, a) \prod_i a_i^{\tilde{f}_i}
	\end{equation}
	最大固有ベクトルを$\ket{\psi(y, a)}$とすると、等式が成立するのは
	\begin{equation}
		\frac{| \braket{y_i | \psi (y, a)} |^2}{a_i} = const
	\end{equation}
	これにより、尤度関数の上限が存在することがわかる。\\

	これまでで、$\hat{\rho}^{(k+1)} = \hat{R} (\hat{\rho}^{(k)}) \hat{\rho}^{(k)}$は$\hat{R}$が対角行列または$\hat{\rho}$が対角行列であれば最尤推定が可能であることが示された。
	しかし、密度行列は非対角項も考慮する必要があり、実験による測定もPOVMとは限らない。
	そこで、密度行列の収束性を利用して$\hat{R} (\hat{\rho}_0) \hat{\rho}_0 \hat{R} (\hat{\rho}_0) = \hat{\rho}_0$となる$\hat{\rho}_0$が存在すると仮定して、規格化定数$\mathcal{N}$を用いて
	\begin{equation}
		\hat{\rho}^{(k+1)} = \mathcal{N} \big[ \hat{R} (\hat{\rho}^{(k)}) \hat{\rho}^{(k)} \hat{R} (\hat{\rho}^{(k)}) \big]
	\end{equation}
	を$\hat{R} \hat{\rho} \hat{R}$アルゴリズムと呼ぶことにする。

	一般に、$\hat{R} \hat{\rho} \hat{R}$アルゴリズムは尤度関数が常に増加するとは言えない。
	そこで、非負の値$\epsilon$を導入することでこの問題を解決する。

	\subsection{Duiluted $\hat{R} \hat{\rho} \hat{R}$アルゴリズム}

	Duiluted $\hat{R} \hat{\rho} \hat{R}$アルゴリズムとは以下の式で表す。
	\begin{equation}
		\hat{\rho}^{(k+1)} = \mathcal{N} \Bigg[ \frac{\hat{I} + \epsilon \hat{R} (\hat{\rho}^{(k)})}{1+\epsilon} \hat{\rho}^{(k)} \frac{\hat{I} + \epsilon \hat{R} (\hat{\rho}^{(k)})}{1+\epsilon} \Bigg]
	\end{equation}
	$\epsilon \ll 1$であれば尤度関数は常に増加する。

	\subsection*{\underline{尤度関数の増加}}

	$\epsilon \ll 1$で$(2.2.19)$式の$\epsilon$の1次までの近似をとると、$(1 + \epsilon)^{-2} \approx 1 - 2\epsilon$より、
	\begin{equation}
		\hat{\rho}^{(k+1)} = \hat{\rho}^{(k)} + \Delta \hat{\rho}
	\end{equation}
	\begin{equation}
		\Delta \hat{\rho} = \epsilon \big( \hat{R} \hat{\rho}^{(k)} + \hat{\rho}^{(k)} \hat{R} -2 \hat{\rho}^{(k)} \big)
	\end{equation}
	また、















\end{document}