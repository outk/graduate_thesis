\documentclass[11pt,a4j,notitlepage]{jreport}

\usepackage{times}
\usepackage[dvipdfmx]{graphicx}	%図を表示するのに必要
\usepackage[dvipdfmx]{color}	%jpgなどを表示するのに必要
\usepackage{amsmath,amssymb}	%数学記号を出すのに必要
\usepackage{setspace}
\usepackage{bm}
\usepackage{braket}	%ブラケットを表示するのに必要
\usepackage{otf}
\usepackage{here}	%図を好きな位置に表示
\usepackage[subrefformat=parens]{subcaption}	%サブキャプション

%PDFの機能(しおり機能、ハイパーリンク機能)が使えるようにする
%しおりの文字化けを防ぐ
\usepackage{atbegshi}
\AtBeginShipoutFirst{\special{pdf:tounicode 90ms-RKSJ-UCS2}}
%hyperrefのverが2007-06-14 6.76i以前の時は↓
%\AtBeginShipoutFirst{\special{pdf:tounicode 90ms-RKSJ-UCS2}}
\usepackage[dvipdfmx,bookmarkstype=toc,colorlinks=true,urlcolor=blue,linkcolor=blue,
citecolor=blue,linktocpage=true,bookmarks=true,setpagesize=false,
pdftitle={量子状態トモグラフィー},
pdfauthor={小林哲也},%
pdfsubject={Bachelor's thesis in 2020}]{hyperref}
\usepackage[numbers,sort]{natbib}
\usepackage{tocbibind}%目次、表一覧、図一覧をしおりに入れる
 
%式、図、表番号の付け方の再定義
\makeatletter
	\renewcommand{\theequation}{%
	\thesection.\arabic{equation}}
	\@addtoreset{equation}{section}
	\def\thefigure{\thesection.\arabic{figure}}
	\@addtoreset{figure}{section}
	\renewcommand{\thetable}{%
	\thesection.\arabic{table}}
	\@addtoreset{table}{section}
\makeatother

%本文と図の余白
\setlength\intextsep{30pt} 
 
\renewcommand\bibname{参考文献}	%関連図書の表示を参考文献に変更
\newcommand{\fig}[1]{図~\ref{#1}}	%図の引用の再定義
\newcommand{\tab}[1]{表~\ref{#1}}	%表の引用の再定義
\newcommand{\eq}[1]{式~\eqref{#1}}	%式の引用の再定義
 
%大きなフォントの定義(表紙用)
\def\HUGE{\fontsize{32pt}{36pt}\selectfont} %\fontsize{フォントの大きさ}{baselineskip}
 
 
%本文
\begin{document}

	%タイトル
	\begin{titlepage}
		\begin{center}\begin{LARGE}
			\vspace{1em}
			{令和元年度}\\
			\vspace{1.5em}
			{卒業研究}\vspace{3em}\\
			\textbf{\HUGE 量子状態トモグラフィー}\\
			\vspace{4em}
			{\LARGE 指導教員}\\
			\vspace{0.8em}
			{\Huge\bf 山本 俊 教授}\\
			\vspace{0.2\vsize}
			{大阪大学 基礎工学部\\ 電子物理科学科 物性物理科学コース\\ 山本研究室\\学籍番号 09D16031}\\
			\vspace{0.8em}
			{\Huge\bf 小林哲也}\\
			\vspace{3em}
			{\Large 2020年3月6日}
		\end{LARGE}\end{center}
	\end{titlepage}

	%目次のページだけローマ数字に設定
	\pagenumbering{roman}

	%目次サブセクションまで表記
	\setcounter{tocdepth}{2}
	\tableofcontents

	\clearpage

	%図目次
	\listoffigures

	%ページ数をリセットしアラビア数字に変更
	\clearpage
	\pagenumbering{arabic}


	\chapter{序論}
	\section{研究の背景と目的}
	近年注目を増している量子情報、量子コンピューティング研究・開発において実験的に生成された量子状態を正確に認識することは極めて重要である。しかしながら、量子的制約により、量子状態を直接観測することは現時点では不可能である。そこで、量子状態トモグラフィーが用いられる。量子状態トモグラフィーとは、同じ量子状態を複数生成し、それぞれを測定基底を変えて測定をすることで得られた観測データから、最尤推定などを用いて真の量子状態を推定することである。本論文では、量子状態トモグラフィーの一般的理論を説明した後、実験的誤りに対する耐性があるアルゴリズムを紹介する。最後に、これらのアルゴリズムを実装し確認する。


	\chapter{理論}
	\section{量子状態トモグラフィーの一般的理論}


	\subsection{密度行列}

	量子状態は密度行列によって表される。密度行列$\hat{\rho}$とは以下の性質を満たす行列である。
	\begin{equation}
		Tr[\hat{\rho}]=1
		\label{eq2.1}
	\end{equation}
	また、密度行列の固有値を$\lambda$とすると、
	\begin{equation}
		0 \leq \lambda \leq 1, \ \ \ \  ^\forall \lambda
		\label{eq2.2}
	\end{equation}
	$d$次元の密度行列を$\hat{\rho}_d$と表記する。



	\subsection{1 qubitトモグラフィー (直交基底)}

	qubitは$2$次元の密度行列で表される。

	1 qubitトモグラフィーのために、パウリ演算子を導入する。パウリ演算子は単位演算子$I$と$SU(2)$の$X,\ Y,\ Z$からなる。
	\begin{equation}
		\begin{aligned}
			I \equiv \hat{\lambda}_0 = \begin{pmatrix} 1 & 0 \\ 0 & 1 \end{pmatrix}&,\ \ X \equiv \hat{\lambda}_1 = \begin{pmatrix} 0 & 1 \\ 1 & 0 \end{pmatrix}\\
			Y \equiv \hat{\lambda}_2 = \begin{pmatrix} 0 & -i \\ i & 0 \end{pmatrix}&,\ \ Z \equiv \hat{\lambda}_3 = \begin{pmatrix} 1 & 0 \\ 0 & -1 \end{pmatrix}
		\end{aligned}
		\label{eq2.3}
	\end{equation}
	これらを用いて、1 qubitの密度行列$\hat{\rho}_2$は次のように表される。
	\begin{equation}
		\hat{\rho}_2 = \frac{1}{2} \sum_{j=0}^3 r_j \hat{\lambda}_j, \ \ \ \ r_j \in Re
		\label{eq2.4}
	\end{equation}
	$SU(2)$はトレースが0なので、密度行列$\hat{\rho}_2$は規格化のために$r_0 = 1$を満たす必要がある。また、ほかのパラメータ$r_{j=1,...,3}$は
	\begin{equation}
		r_1^2 + r_2^2 + r_3^2 \leq 1
		\label{eq2.5}
	\end{equation}
	のみ満たす。また、$r_j$は$r_j = Tr[\hat{\rho}_2 \hat{\lambda}_j]$で得られる。\\
	
	したがって、1 qubitの密度行列は
	\begin{equation}
		\hat{\rho}_2 = \frac{1}{2} \begin{pmatrix}
			1 + \braket{\hat{\lambda}_3} &
			\braket{\hat{\lambda}_1} - i \braket{\hat{\lambda}_2} \\
			\braket{\hat{\lambda}_1} + i \braket{\hat{\lambda}_2} &
			1 - \braket{\hat{\lambda}_3}
			\end{pmatrix}
	\end{equation}
	で表される。
	
	上式から、1 qubitの密度行列は3つの測定だけで求まりそうだが、実験的には4つ目の基底$\hat{\lambda}_0$の測定によって密度行列の規格化が必要である。また、$\langle \hat{\lambda}_j \rangle$の値によっては、$(2.1.2)$式を満たさないことがある。したがって、最尤推定などを用いて物理的に意味のある密度行列を見つける必要がある。

	$SU(2)$生成子は必ず物理的状態を示すとは限らないが、$\hat{\lambda}_{0,...,3}$は常に物理的に意味のある状態の密度行列の線形和で表すことができる。例えば、パウリ演算子は光学系での測定としては物理的意味がない。しかし、光子の測定を偏向基底で行う際に以下のようなものが用いられる。
	\begin{equation}
		\begin{aligned}
			\ket{H} \bra{H} = \frac{1}{2}[\hat{\lambda}_0 + \hat{\lambda}_3] \ket{V} \bra{V} = \frac{1}{2}[\hat{\lambda}_0 - \hat{\lambda}_3]\\
			\ket{D} \bra{D} = \frac{1}{2}[\hat{\lambda}_0 + \hat{\lambda}_1] \ket{L} \bra{L} = \frac{1}{2}[\hat{\lambda}_0 + \hat{\lambda}_2]
		\end{aligned}
	\end{equation}
	ここで、$\ket{H} = \ket{0}, \ \ket{V} = \ket{1}, \ \ket{D} = (\ket{0} + \ket{1}) / \sqrt{2} , \ \ket{L} = (\ket{0} + i\ket{1}) / \sqrt{2}$

	このようにどの直交した測定基底を選んでも他のいくつかの測定演算子$\hat{\Pi}_k$を用いて$\hat{\lambda}_j = \frac{1}{2} \sum_{k} a_{jk} \hat{\Pi}_k$と表すことができる。そして、トモグラフィーは測定結果$a_{jk} = \langle \hat{\Pi}_k \rangle = Tr[\hat{\rho}_2 \hat{\Pi}_k]$を測定することで行われる。

	\subsection{1 qubitトモグラフィー (非直交基底)}

	実際には、測定は測定装置側の基底を偏向せずに量子状態を変化させて測定する。そのため、1 qubitの状態を測定基底に合わせて、$| 0 \rangle + | 1 \rangle$や$| 0 \rangle + i | 1 \rangle$から$| 0 \rangle$へ量子状態を変化させることが難しい場合がある。その場合、測定基底を$| 0 \rangle$と
	\begin{equation}
		\begin{aligned}
			\ket{\theta_+} &= \frac{1}{\sqrt{2}} [\cos \theta \ket{0} + \sin \theta \ket{1}]\\
			\ket{\varphi_+} &= \frac{1}{\sqrt{2}} [\cos \varphi \ket{0} + i \sin \varphi \ket{1}]
		\end{aligned}
	\end{equation}
	とすることができる。$\theta, \varphi$は小さい値でもよい。つまり、1 qubitトモグラフィーはある測定基底と少しの摂動があれば行える。実験系によってはこれは重要になることがある。

	任意の基底$\ket{\psi_\nu}$での測定は射影子$\hat{\lambda}_\nu = \ket{\psi_\nu} \bra{\psi_\nu}$で表され、これらの基底による観測回数$n_\nu$は
	\begin{equation}
		n_\nu = \mathcal{N} \braket{\psi_\nu | \hat{\rho} | \psi_\nu}
	\end{equation}
	で表される。($\mathcal{N}$は定数)

	\subsection{Quditへの拡張}

	まず、$SU(d)$を準備する。($d$次元の$SU$群)

	$d$次元の要素行列$\{ e_j^k|k,j=1,...,d \}$は
	\begin{equation}
		\big( e_j^k \big) _{\mu \nu} = \delta_{\nu j} \delta_{\mu k}, \ \ \ \ 1 \leq \nu, \mu \leq d
	\end{equation}
	で表され、$\mu$行$\nu$列目の要素が1で他の要素すべてが0である行列である。

	これらの行列は交換関係を満たす。
	\begin{equation}
		\big[ e_j^i, e_l^k \big] = \delta_{kj} e_l^i - \delta_{il} e_j^k
	\end{equation}
	$d(d-1)$個のトレースが0の行列が存在する。
	\begin{equation}
		\begin{aligned}
			\Theta_j^k = e_j^k + e_k^j, \ \ \beta_j^k = -i \big( e_j^k - e_k^j \big) \ \ \ \ \ \ \ \ 1 \leq k < j \leq d
		\end{aligned}
	\end{equation}
	これらは$SU(d)$群の非対角生成子である。

	対角生成子として残り$d-1$個のトレースが0の行列を
	\begin{equation}
		\eta_r^r = \sqrt{\frac{2}{r(r-1)}} \Biggl[ \sum_{j=1}^r e_j^j - r e_{r+1}^{r+1} \Biggr]
	\end{equation}
	とすると、これで$d^2-1$個の生成子が得られる。\\
	ここで$\lambda$行列を次のように定義する。
	\begin{equation}
		\begin{aligned}
			\hat{\lambda}_{(j-1)^2 + 2(k-1)} &= \Theta_j^k\\
			\hat{\lambda}_{(j-1)^2 + 2k-1} &= \beta_j^k\\
			\hat{\lambda}_{j^2-1} &= \eta_{j-1}^{k-1}
		\end{aligned}
	\end{equation}
	$d$次元に拡張してもこれらの形式は完全エルミート演算子基底である。つまり、$\hat{\lambda}$がエルミートかつ次式を満たす。
	\begin{equation}
		\sum_{j=0}^{d^2 - 1} \hat{\lambda}_j = \hat{1}
	\end{equation}

	$d$次元に拡張しても$(2.1.4)$式はそのまま適用できる。つまり、密度行列$\hat{\rho}_d$は生成子の線形結合で表される。
	\begin{equation}
		\hat{\rho}_d = \frac{1}{d} \sum_{j=0}^{d^2 - 1} r_j \hat{\lambda}_j
	\end{equation}
	これは1 quditの密度行列である。規格化のために係数$r_0$は1とし、$Tr \big[ \hat{\rho}_d^2 \big] \leq 1$より$\sum_{j=1}^{d^2 - 1} r_j^2 \leq d(d-1)/2$を満たす。

	\subsection{Multi quditsへの拡張}

	Multi qubitsでは、演算子のヒルベルト空間を規格化された単位行列$\hat{\lambda}_0$を含んだ$SU(2)$を生成子のテンソル積で定義する。
	\begin{equation}
		SU(2) \otimes SU(2) \otimes ・・・ \otimes SU(2)
	\end{equation}

	2 quditsでは$d^2$の次元を持った密度行列$\hat{\rho}_{2 d}$は同様に拡張できる。\\
	$\hat{\lambda}_0$を含む$\hat{\lambda}$行列のテンソル積$\hat{\lambda}_{j 1} \otimes \hat{\lambda}_{j 2}$のすべての組はそれぞれ線形独立なので、$\hat{\rho}_{2 d}$は次のように表される。
	\begin{equation}
		\hat{\rho}_{2 d} = \frac{1}{d^2} \sum_{j1,j2=0}^{d^2 - 1} r_{j 1, j 2} \hat{\lambda}_{j 1} \otimes \hat{\lambda}_{j 2}
	\end{equation}
	同様に、n quditsでは
	\begin{equation}
		\hat{\rho}_{n d} = \frac{1}{d^n} \sum_{j1,...,jn=0}^{d^2 - 1} r_{j 1,...,j n} \hat{\lambda}_{j 1} \otimes ・・・ \otimes \hat{\lambda}_{j n}
	\end{equation}

	\subsection{密度行列の再構成}

	簡単のために$\hat{\Gamma}_\nu = \hat{\lambda}_{j 1} \otimes ・・・ \otimes \hat{\lambda}_{j n}$とすると、密度行列は
	\begin{equation}
		\hat{\rho}_{n d} = \sum_{\nu = 0}^{d^n - 1} \tilde{r}_\nu \hat{\Gamma}_\nu
	\end{equation}
	で表される。$\tilde{r}_\nu$は$d^n$要素あるベクトルの$\nu$番目の要素で
	\begin{equation}
		\tilde{r}_\nu = Tr \big[ \hat{\Gamma}_\nu \hat{\rho}_{n d} \big]
	\end{equation}
	これを$(2.1.9)$式に代入して
	\begin{equation}
		n_\nu = \mathcal{N} \sum_{\mu = 0}^{d^n - 1} B_{\nu, \mu} \tilde{r}_\nu
	\end{equation}
	ここで、$B_{\nu, \mu}$は$d^n \times d^n$行列の$\nu$行$\mu$列番目の要素で
	\begin{equation}
		B_{\nu, \mu} = \braket{\psi_\nu | \hat{\Gamma}_\mu | \psi_\nu}
	\end{equation}
	$B_{\nu, \mu}$が可逆行列であれば
	\begin{equation}
		\tilde{r}_\nu = \mathcal{N}^{-1} \sum_{\nu = 0}^{d^n - 1} \hat{M}_\nu n_\nu = \sum_{\nu = 0}^{d^n - 1} \hat{M}_\nu s_\nu
	\end{equation}
	ここで、$\hat{M}_\nu$は$d \times d$行列で
	\begin{equation}
		\hat{M}_\nu = \sum_{\nu = 0}^{d^n - 1} \big( B^{-1} \big)_{\nu, \mu} \hat{\Gamma}_\nu
	\end{equation}
	$\hat{M}_\nu$の性質から
	\begin{equation}
		\sum_\nu Tr \big[ \hat{M}_\nu \big] \ket{\psi_\nu} \bra{\psi_\nu} \hat{\rho}_{n d} = \hat{\rho}_{n d}
	\end{equation}
	両辺でトレースをとると
	\begin{equation}
		\sum_\nu Tr \big[ \hat{M}_\nu \big] n_\nu = \mathcal{N}
	\end{equation}
	したがって、任意の密度行列$\hat{\rho}_{n d}$は次のように再構成される。
	\begin{equation}
		\hat{\rho}_{n d} = \frac{\sum_\nu \hat{M}_\nu n_\nu}{\sum_\nu Tr \big[ \hat{M}_\nu \big] n_\nu} 
	\end{equation}


	\section{最尤推定}












\end{document}